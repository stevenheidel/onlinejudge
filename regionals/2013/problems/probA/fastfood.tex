\problemname{Fast Food Prizes}

\illustration{0.38}{pieces}{}%
Around regional contest time, the Canadian branch of a popular fast
food restaurant usually runs a game to promote its business.  Certain
food items provide stickers, and certain collection of different
stickers can be converted to cash prizes.  If a prize requires sticker
types $T_1, T_2, \ldots, T_k$, then you can claim the prize if you
have 1 sticker of each type $T_1, T_2, \ldots, T_k$.  Each sticker
can only be used to claim one prize.  However, you may claim a prize
multiple times if you have multiple stickers of the same type.  No two
prizes will require the same type of stickers.  There may be some
stickers that cannot be used to claim a cash prize (e.g. a sticker
for a free milkshake).

On your road trip to the regional contest, your coach forced you to
eat at this restaurant and collected all the stickers together.  How
much cash can your coach claim?

\section*{Input}
The input consists of multiple test cases. The first line of input
is a single integer, not more than $1000$, indicating the number 
of test cases to follow.  
Each case starts with a line containing two integers $n$ ($1
\leq n \leq 10$) and $m$ ($1 \leq m \leq 30$), where $n$ is the number of different
types of prizes, and $m$ is the number of different types of stickers
(the types are labelled $1, 2, \ldots, m$).  The next $n$ lines
specify the prizes.  Each of these lines starts with an integer $k$ ($1
\leq k \leq m$) specifying the number of sticker types required to
claim the prize.  This is followed by $k$ integers specifying the
types of the stickers required.  The final integer on each line is the
(positive) cash value of the prize (at most 1,000,000).  The last line of each
case gives $m$ nonnegative integers, with the $i$th integer giving
the number of stickers of type $i$ your coach has collected. There are no
more than 100 stickers of each type.

\section*{Output}
For each case, display on a single line the total value of
the cash prizes that can be claimed.

